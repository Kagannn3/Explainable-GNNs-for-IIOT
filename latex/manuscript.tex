\documentclass[journal]{IEEEtran}
\usepackage{cite}
\usepackage{graphicx}
\usepackage{amsmath,amssymb}
\usepackage{booktabs}
\usepackage{url}

\begin{document}

\title{Explainable Graph Neural Networks for Predictive Maintenance in Industrial IoT}

\author{Oğuz Kağan Koçak}
\date{\today}

\thanks{This work was supported by ...}% <-this % stops a space
\thanks{Oğuz Kağan Koçak is with Dept.\ of Computer Engineering, Antalya, Turkey.}%

}

\maketitle

\begin{abstract}
% 150–250 words summarizing motivation, approach (federated non‐IID GNNs + XAI), key results (RMSE improvements, fidelity/simplicity gains), and implications.
\end{abstract}

\begin{IEEEkeywords}
Predictive Maintenance, Industrial IoT, Graph Neural Networks, Explainable AI, Federated Learning.
\end{IEEEkeywords}

\IEEEpeerreviewmaketitle

\section{Introduction}
\label{sec:intro}
% - Context: predictive maintenance in IIoT
% - Limitations of existing sequential models
% - Potential of GNNs to model sensor relations
% - Need for explainability (safety, trust)
% - Contributions:
%   1) Federated non-IID GCN/GAT pipeline
%   2) Integrated XAI (GNNExplainer, IG)
%   3) Empirical results on CMAPSS
%   4) Fidelity & simplicity analysis

\section{Related Work}
\label{sec:related}
% (Draw from Phase 1 survey; cite ~15 works)
% 2–3 paragraphs:
%  - IIoT predictive maintenance methods (classical ML, LSTM)
%  - GNNs in IoT (GCN, GAT applications)
%  - Explainability in GNNs (GNNExplainer, SHAP, Captum)
%  - Federated/non-IID considerations
% Concluding gap statement.

\section{Methodology}
\label{sec:method}
\subsection{Data and Federated Non-IID Setup}
% - CMAPSS FD003 description
% - Sliding-window framing (size=30, stride=5)
% - Operating-condition–based non-IID split into 3 clients
% - Graph construction via Pearson correlation threshold $\tau=0.8$

\subsection{Baseline Model: LSTM}
% - Architecture details
% - Input/output dimensions
% - Training hyperparameters

\subsection{Graph Neural Network Models}
\subsubsection{GCN Regressor}
% - Two-layer GCN, global pooling, regression head

\subsubsection{GAT Regressor}
% - Two-layer multi-head GAT, pooling, regression head

\subsection{Explainability Framework}
% - Integrated Gradients for LSTM
% - GNNExplainer for GCN/GAT
% - Metrics: fidelity, simplicity

\section{Experiments}
\label{sec:experiments}
\subsection{Experimental Setup}
% - Hyperparameter grid (hidden $\in\{32,64\}$, lr $\in\{10^{-3},5\times10^{-4}\}$, epochs=10)
% - Evaluation metrics: MAE, RMSE

\subsection{Performance Results}
\begin{table}[!t]
  \caption{Average MAE and RMSE across clients}
  \label{tab:performance}
  \centering
  \begin{tabular}{lcc}
    \toprule
    Model & MAE & RMSE \\
    \midrule
    LSTM  & 9.12 & 12.30 \\
    GCN   & 7.45 & 10.10 \\
    GAT   & \textbf{6.98} & \textbf{9.65} \\
    \bottomrule
  \end{tabular}
\end{table}

\begin{figure}[!t]
  \centering
  \includegraphics[width=\columnwidth]{figures/fig2_performance_summary.png}
  \caption{MAE and RMSE comparison across models and clients.}
  \label{fig:perf}
\end{figure}

\subsection{Communication vs.\ Performance}
\begin{figure}[!t]
  \centering
  \includegraphics[width=\columnwidth]{figures/comm_vs_rmse.png}
  \caption{Trade-off between communication cost (bytes) and RMSE for each model.}
  \label{fig:comm}
\end{figure}

\section{Explainability Analysis}
\label{sec:xai}
\subsection{Fidelity and Simplicity Metrics}
\begin{table}[!t]
  \caption{Median Fidelity and Simplicity by Model}
  \label{tab:xai_metrics}
  \centering
  \begin{tabular}{lcc}
    \toprule
    Model & Fidelity & Simplicity \\
    \midrule
    GCN & 0.82 & 0.40 \\
    GAT & \textbf{0.85} & \textbf{0.35} \\
    \bottomrule
  \end{tabular}
\end{table}

\subsection{Case Study: Subgraph Explanations}
\begin{figure}[!t]
  \centering
  \includegraphics[width=\columnwidth]{figures/fig3_xai_case_study.png}
  \caption{GAT subgraph explanation for a failure-case window (true RUL vs.\ predicted).}
  \label{fig:case}
\end{figure}

\section{Discussion}
\label{sec:discussion}
% - Interpretation of performance gains
% - Value of explainability (trust, root-cause insights)
% - Impact of non-IID federation on model consistency
% - Limitations and potential extensions (dynamic graphs, larger-scale deployment)

\section{Conclusion}
\label{sec:conclusion}
% - Recap contributions and findings
% - Future work: adaptive federated aggregation, real-time dashboard deployment, SECOM evaluation

\appendices
\section{Hyperparameter Details}
% Table of all grid configurations

\section*{Acknowledgments}
% Funding, colleagues, data providers.

\bibliographystyle{IEEEtran}
\bibliography{bibliography}

\end{document}
